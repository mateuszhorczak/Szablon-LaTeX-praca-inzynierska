% Preamble
\documentclass[12pt, a4paper]{report}

% Packages
\usepackage{amsmath}
\usepackage{graphicx}
\usepackage[polish]{babel}
\usepackage[T1]{fontenc}
\usepackage{newtxtext,newtxmath} % wspiera polsie znaki lepiej od mathptmx
\usepackage{polski}
\usepackage[utf8]{inputenc}
\usepackage{tabularx}
\usepackage{listings}
\usepackage{xcolor}
\usepackage{float}
\usepackage{url}

\usepackage{tocloft}
\renewcommand{\cftchapleader}{\cftdotfill{\cftdotsep}}
\renewcommand{\cftsecleader}{\cftdotfill{\cftdotsep}}

\renewcommand{\listfigurename}{Spis rysunków}
\renewcommand{\listtablename}{Spis tabel}
\renewcommand{\lstlistlistingname}{Spis listingów}

\usepackage[a4paper,top=25mm,right=20mm,bottom=25mm,left=35mm]{geometry} % marginesy
\setlength{\parindent}{1cm} % Wcięcia akapitu

\graphicspath{{images/}}

\definecolor{codegreen}{rgb}{0,0.6,0}
\definecolor{codegray}{rgb}{0.5,0.5,0.5}
\definecolor{codepurple}{rgb}{0.58,0,0.82}
\definecolor{backcolour}{rgb}{0.95,0.95,0.92}

% Przykładowe zdefiniowanie składni do listingów w LaTeX dla TypeScript
\lstdefinelanguage{TypeScript}{
  keywords={break, case, catch, continue, debugger, default, delete, do, else, false, finally, for, function, if, in, instanceof, new, null, return, switch, this, throw, true, try, typeof, var, void, while, with, let, const, async, await},
  ndkeywords={class, export, boolean, throw, implements, import, interface, type, this, extends, public, private, protected, readonly, enum, namespace, declare},
  morecomment=[l]{//},
  morecomment=[s]{/*}{*/},
  morestring=[b]',
  morestring=[b]",
  morestring=[b]`,
  keywordstyle=\color{blue}\bfseries,
  ndkeywordstyle=\color{darkgray}\bfseries,
  identifierstyle=\color{black},
  commentstyle=\color{purple}\ttfamily,
  stringstyle=\color{red}\ttfamily,
  sensitive=true
}

\lstset{
   language=TypeScript,
   backgroundcolor=\color{lightgray},
   extendedchars=true,
   basicstyle=\footnotesize\ttfamily,
   showstringspaces=false,
   showspaces=false,
   numbers=left,
   numberstyle=\footnotesize,
   numbersep=9pt,
   tabsize=2,
   breaklines=true,
   showtabs=false,
   captionpos=b
}

% Przykładowe zdefiniowanie składni do listingów w LaTeX dla Vue
\lstdefinelanguage{Vue}{
  keywords={break, case, catch, continue, debugger, default, delete, do, else, false, finally, for, function, if, in, instanceof, new, null, return, switch, this, throw, true, try, typeof, var, void, while, with, let, const, async, await},
  ndkeywords={class, export, boolean, throw, implements, import, this, extends, public, private, protected, readonly, v-bind, v-if, v-else, v-else-if, v-for, v-model, v-on, v-show, v-slot, v-pre, v-cloak, template, component, props, data, methods, computed, watch, emits, setup, template, NuxtLayout, NuxtPage},
  morecomment=[l]{//},
  morecomment=[s]{/*}{*/},
  morecomment=[s]{<!--}{-->},
  morestring=[b]',
  morestring=[b]",
  morestring=[b]`,
  keywordstyle=\color{blue}\bfseries,
  ndkeywordstyle=\color{darkgray}\bfseries,
  identifierstyle=\color{black},
  commentstyle=\color{purple}\ttfamily,
  stringstyle=\color{red}\ttfamily,
  sensitive=true
}

\lstset{
   language=Vue,
   backgroundcolor=\color{lightgray},
   extendedchars=true,
   basicstyle=\footnotesize\ttfamily,
   showstringspaces=false,
   showspaces=false,
   numbers=left,
   numberstyle=\footnotesize,
   numbersep=9pt,
   tabsize=2,
   breaklines=true,
   showtabs=false,
   captionpos=b
}


\usepackage{color}
\usepackage{hyperref}
\hypersetup{
    % linktocpage, % option to only link the page numbers and not the entire table of contents; when links are being coloured the default behaviour can be a bit overwhelming.
    colorlinks=true, %set true if you want colored links
    linktoc=all,     %set to all if you want both sections and subsections linked
    linkcolor=black,  %choose some color if you want links to stand out
    citecolor=black,
    urlcolor=black,
}

% Document
\begin{document}
\addtocontents{toc}{\protect\thispagestyle{empty}}
\tableofcontents
\thispagestyle{empty}
\newpage
\setcounter{page}{1}


    \chapter{Wprowadzenie}\label{ch:wprowadzenie}
    \hspace{1cm}bla bla bla



    \chapter{Analiza wymagań}\label{ch:analiza-wymagan}
    \hspace{1cm}Lorem ipsum

lorem ipsum bla bla

\section{Baza technologiczna}\label{sec:baza-technologiczna}
% Opis bazy technologicznej

bla bla

\section{Baza danych}\label{sec:baza-danych}
% Opis bazy danych

bla bla

\section{Implementacja}\label{sec:implementacja}
% Opis implementacji

bla bla

\section{Testy}\label{sec:testy}

bla bla




    \chapter{Wybór technologii}\label{ch:wybor-technologii}
    \hspace{1cm}lorem ipsum

bla bla

\section{Baza technologiczna}\label{sec:baza-technologiczna}
% Opis bazy technologicznej

bla bla

\section{Baza danych}\label{sec:baza-danych}
% Opis bazy danych

bla bla

\section{Implementacja}\label{sec:implementacja}
% Opis implementacji

bla bla




    \chapter{Realizacja}\label{ch:realizacja}
    \hspace{1cm}lorem ipsum

bla bla

\section{Baza technologiczna}\label{sec:baza-technologiczna}
% Opis bazy technologicznej

bla bla

\section{Baza danych}\label{sec:baza-danych}
% Opis bazy danych

bla bla

\section{Implementacja}\label{sec:implementacja}
% Opis implementacji

bla bla

\section{Testy}\label{sec:testy}

bla bla

\section{Wdrożenie}\label{sec:wdrozenie}

bla bla




    \chapter{Prezentacja aplikacji}\label{ch:prezentacja-aplikacji}
    \hspace{1cm}bla bla bla bla



    \chapter{Podsumowanie}\label{ch:podsumowanie}
    % W podsumowaniu należy przede wszystkim odpowiedzieć na pytanie, czy i w jakim stopniu cel pracy, określony we wstępie, został osiągnięty.

% Podsumowanie powinno także przedstawiać wnioski, które wynikły w trakcie realizacji pracy lub stanowią efekt syntezy osiągniętych wyników. Wszystkie wnioski powinny 	wynikać z treści pracy.

% Podsumowanie może także identyfikować kierunki ewentualnej kontynuacji pracy i pokreślać sposób przyszłego wykorzystania wyników.

\hspace{1cm}bla bla


    \newpage
    \renewcommand{\thepage}{}

    \bibliography{main}
    \bibliographystyle{plain}
    \addcontentsline{toc}{chapter}{Bibliografia}

    \thispagestyle{empty}
    \listoffigures
    \listoftables
    \lstlistoflistings
    \clearpage
    \pagenumbering{arabic}
\end{document}
